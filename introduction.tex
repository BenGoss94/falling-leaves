\section{Introduction}
When falling leaves are observed it can be seen that they can display a wide range of dynamics. This range of dynamics can be difficult to model, however it could be possible to make shortcuts to simplify the model while maintaining the complete range of motions. A simple model is required in the application of the model to computer animation when thousands of leaves are simulated. 
\\
\\
\noindent Animation is a key part of the media industry. As computer-generated imagery (CGI) is used more frequently in films and video games, demand for more realistic looking scenes increases. In 2013, the video game industry accounted for \$21.53 billion of consumer spending, with \$15.39 billion spent by consumers on the software and content\textsuperscript{\cite{videogameindustry}}. As the industry grows, consumers look for content that looks more realistic whilst still running on the same hardware. It becomes more important to map real life motion realistically and efficiently into virtual environments. 
\\
\\
\noindent There are many possible models to describe a falling leaf or similar body such as the model proposed by Andersen, Pesavent and Wang which considers a model derived from experiments and simulations\textsuperscript{\cite{andersen2005analysis}}. In this investigation the model proposed by Tanabe and Kaneko will be considered\textsuperscript{\cite{tanabe1994behavior}}.
\\
\\
The influence of the parameters on the behaviour of the falling leaf will be investigated. This will allow the most important parameters to be identified, with the goal of minimising the number of parameters to create a simpler model to be used in a computer animation. The initial conditions will also be considered to see how they influence the motion of the leaf. The system will be analysed to find any equilibria. If any are found they will be analysed to determine their stability as well as the possibility of bifurcations. 
\\
